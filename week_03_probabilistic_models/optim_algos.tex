\documentclass[10pt]{article}
\usepackage{amsmath, amssymb, geometry, enumitem}
\geometry{margin=1in}
\title{\textbf{Summary of Optimization Algorithms in Deep Learning}}
\author{Abdul Samad Khan}
\date{}

\begin{document}

\maketitle

\section*{Overview}
This note summarizes key optimization methods relevant to training deep learning models, including those applicable to volatility forecasting tasks using Transformer-GNN architectures.

\section*{1. Stochastic Gradient Descent (SGD)}
SGD updates model parameters using noisy gradients computed on mini-batches:
\[
\theta_{t+1} = \theta_t - \eta \nabla_\theta L(\theta_t)
\]
While simple, SGD is sensitive to learning rate and often slow to converge.

\section*{2. SGD with Momentum}
Momentum improves SGD by adding a velocity term:
\begin{align*}
v_t &= \gamma v_{t-1} + \eta \nabla_\theta L(\theta_t) \\
\theta_{t+1} &= \theta_t - v_t
\end{align*}
Here, $\gamma$ (typically 0.9) controls the contribution of past gradients, smoothing updates and accelerating convergence.

\section*{3. Nesterov Accelerated Gradient (NAG)}
NAG adds a \emph{look-ahead} step before computing the gradient:
\begin{align*}
v_t &= \gamma v_{t-1} + \eta \nabla_\theta L(\theta_t - \gamma v_{t-1}) \\
\theta_{t+1} &= \theta_t - v_t
\end{align*}
This anticipatory behavior improves stability and convergence, especially in non-convex landscapes.

\section*{4. Adagrad}
Adagrad adapts the learning rate per parameter based on the sum of squared past gradients:
\[
\theta_{t+1} = \theta_t - \frac{\eta}{\sqrt{G_t + \epsilon}} \nabla_\theta L(\theta_t)
\]
Good for sparse data, but learning rates shrink too much over time.

\section*{5. RMSprop}
RMSprop fixes Adagrad’s decay problem using an exponential moving average:
\begin{align*}
E[g^2]_t &= \gamma E[g^2]_{t-1} + (1 - \gamma) g_t^2 \\
\theta_{t+1} &= \theta_t - \frac{\eta}{\sqrt{E[g^2]_t + \epsilon}} g_t
\end{align*}
Performs well on non-stationary problems; was designed for RNNs.

\section*{6. Adam}
Adam combines RMSprop and Momentum:
\begin{align*}
m_t &= \beta_1 m_{t-1} + (1 - \beta_1) g_t \\
v_t &= \beta_2 v_{t-1} + (1 - \beta_2) g_t^2 \\
\hat{m}_t &= \frac{m_t}{1 - \beta_1^t},\quad \hat{v}_t = \frac{v_t}{1 - \beta_2^t} \\
\theta_{t+1} &= \theta_t - \eta \frac{\hat{m}_t}{\sqrt{\hat{v}_t} + \epsilon}
\end{align*}
Default optimizer for most deep learning tasks due to robustness and speed.

\section*{7. BFGS (Quasi-Newton)}
BFGS approximates second-order information without computing the Hessian:
\[
\theta_{t+1} = \theta_t - H_t^{-1} \nabla_\theta L(\theta_t)
\]
Too computationally expensive for deep networks but useful in convex optimization problems (e.g., classical ML or econometric models).

\section*{Recommendation}
For training large-scale models like Transformer-GNN hybrids in financial time-series:
\begin{itemize}[leftmargin=*]
  \item Use \textbf{Adam or AdamW} as default.
  \item Use \textbf{SGD + Nesterov Momentum} for better generalization on large datasets.
  \item Avoid Adagrad/BFGS unless solving niche or small-scale convex problems.
\end{itemize}

\end{document}
